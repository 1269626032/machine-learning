%% \documentclass{ctexart}

%% \usepackage{amsmath}
%% \usepackage{amsthm}
%% \usepackage{amsfonts}

%% \title{Some Mathematical Aspects of Machine Learling}
%% \author{王逸舟}
%% \date{2018年12月31日}

%% \begin{document}

%% \maketitle
%% \tableofcontents

%% \section{Optimizing Neural Networks in the Equivalence Class Space}

%% \subsection{概念}
%% \subsubsection{机器学习}
\section{机器学习}
Langley(1996) 定义的机器学习是“机器学习是一门人工智能的科学,该领域的主要研究对象是人工智能,特别是如何在经验学习中改善具体算法的性能”。(Machine learning is a science of the artificial. The field's main objects of study are artifacts, specifically algorithms that improve their performance with experience.')

Tom Mitchell的机器学习(1997)对信息论中的一些概念有详细的解释,其中定义机器学习时提到,“机器学习是对能通过经验自动改进的计算机算法的研究”。(Machine Learning is the study of computer algorithms that improve automatically through experience.)

Alpaydin(2004)同时提出自己对机器学习的定义,“机器学习是用数据或以往的经验,以此优化计算机程序的性能标准。”(Machine learning is programming computers to optimize a performance criterion using example data or past experience.)
\subsection{十大经典机器学习算法}

\subsubsection{决策树}
根据一些 feature(特征) 进行分类,每个节点提一个问题,通过判断,将数据分为两类,再继续提问。这些问题是根据已有数据学习出来的,再投入新数据的时候,就可以根据这棵树上的问题,将数据划分到合适的叶子上。
\subsubsection{随机森林}
在源数据中随机选取数据,组成几个子集:
S矩阵是源数据,有1-N条数据,A、B、C 是feature,最后一列C是类别:
由S随机生成M个子矩阵:
这M个子集得到 M 个决策树:将新数据投入到这M个树中,得到M个分类结果,计数看预测成哪一类的数目最多,就将此类别作为最后的预测结果。
\subsubsection{逻辑回归}
当预测目标是概率这样的,值域需要满足大于等于0,小于等于1的,这个时候单纯的线性模型是做不到的,因为在定义域不在某个范围之内时,值域也超出了规定区间。
那么怎么得到这样的模型呢?

这个模型需要满足两个条件 “大于等于0”,“小于等于1” 。大于等于0 的模型可以选择绝对值,平方值,这里用指数函数,一定大于0;小于等于1 用除法,分子是自己,分母是自身加上1,那一定是小于1的了。

(1)\(p \geq 0\)

\[p=exp(\beta_0+\beta_1 age)\]

(2)\(p\leq 1\)

\[p=\frac{exp(\beta_0+\beta_1age)}{exp(\beta_0+\beta_1age)+1}\]
于是就得到

\[\ln \left(\frac{p}{1-p}\right)=\beta_0+\beta_1age\]
\subsubsection{支持向量机}
要将两类分开,想要得到一个超平面,最优的超平面是到两类的 margin 达到最大,margin就是超平面与离它最近一点的距离

将这个超平面表示成一个线性方程,在线上方的一类,都大于等于1,另一类小于等于-1

\(g(x)\geq 1,\forall x \in class 1\)\\
\(g(x)\leq 1,\forall x \in class2\)
\subsubsection{朴素贝叶斯}
\subsubsection{K近邻算法}
给一个新的数据时,离它最近的 k 个点中,哪个类别多,这个数据就属于哪一类。
\subsubsection{K均值算法}
先要将一组数据,分为三类,粉色数值大,黄色数值小 。最开始先初始化,这里面选了最简单的 3,2,1 作为各类的初始值 。剩下的数据里,每个都与三个初始值计算距离,然后归类到离它最近的初始值所在类别。

分好类后,计算每一类的平均值,作为新一轮的中心点

几轮之后,分组不再变化了,就可以停止了:
\subsubsection{Adaboost}
Adaboost 是 Boosting 的方法之一。Boosting就是把若干个分类效果并不好的分类器综合起来考虑,会得到一个效果比较好的分类器。

下图,左右两个决策树,单个看是效果不怎么好的,但是把同样的数据投入进去,把两个结果加起来考虑,就会增加可信度。
\subsubsection{神经网络}
Neural Networks适合一个input可能落入至少两个类别里:NN由若干层神经元,和它们之间的联系组成。 第一层是input层,最后一层是output层。在hidden层和output层都有自己的classifier。

input输入到网络中,被激活,计算的分数被传递到下一层,激活后面的神经层,最后output层的节点上的分数代表属于各类的分数,下图例子得到分类结果为class 1;同样的input被传输到不同的节点上,之所以会得到不同的结果是因为各自节点有不同的weights 和bias,这也就是forward propagation。
\subsubsection{马尔可夫}

Markov Chains由state(状态)和transitions(转移)组成。例子,根据这一句话 ‘the quick brown fox jumps over the lazy dog’,要得到markov chains。

步骤,先给每一个单词设定成一个状态,然后计算状态间转换的概率。

这是一句话计算出来的概率,当你用大量文本去做统计的时候,会得到更大的状态转移矩阵,例如the后面可以连接的单词,及相应的概率。

